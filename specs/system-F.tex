\documentclass[12pt]{article}
\usepackage[letterpaper,margin=0.5in]{geometry}
\usepackage{amsmath,amssymb,latexsym}
\usepackage{float,dsfont,multirow,underscore}
\usepackage{indentfirst,multicol,mathtools,xcolor}
\usepackage{upquote}
\usepackage{lmodern}

\date{\vspace{-5ex}}
\author{\vspace{-5ex}}
\title{Polylamb Language Specification}

\newcommand{\gm}[1]{$#1$} %Grammars are italicized
\newcommand{\kwt}[1]{\textbf{\texttt{#1}}} % Keywords in Text mode
\newcommand{\kwm}[1]{$\pmb{#1}$} % Keywords in Math mode
\newcommand{\row}[3]{&\gm{#1} &:= &#2 &#3\\}
\newcommand{\newrow}[2]{& & &#1 &#2\\}

% Uniform column widths for table
\def\ColOne{1.0cm}
\def\ColTwo{0.5cm}
\def\ColThree{8.0cm}
\def\ColFour{6.0cm}

\begin{document}
\maketitle

\section{Syntax}
\subsection{Expressions}
\begin{figure}[h]
  \centering
  \begin{tabular}{l p{\ColOne} p{\ColTwo} p{\ColThree} p{\ColFour}}
    \hline
    \row{e}{\gm{l}}{literals}
    \newrow{\gm{v}}{variables (lowercase)}
    \newrow{\kwt{(}\gm{e}\kwt{)}}{parenthesized}
    \newrow{\gm{e}\kwt{[}\gm{\tau}\kwt{]}}{type application}
    \newrow{\gm{e_1} \gm{e_2}}{application}
    \newrow{\gm{e_1} \gm{op} \gm{e_2}}{binary operation}
    \newrow{\kwt{(}\gm{e_1}, \gm{\dots}, \gm{e_n}\kwt{)}}{$n$-tuples, $n\geq 2$}
    \newrow{\kwm{\lambda} \gm{v}\kwt{:}\gm{\tau}\kwt{.} \gm{e}}{lambda abstraction\footnotemark[1]}
    \newrow{\kwm{\Lambda} \gm{t}\kwt{.} \gm{e}}{type abstraction\footnotemark[2]}
    \newrow{\kwt{let} \gm{p} \kwt{=} \gm{e_1} \kwt{in} \gm{e_2}}{let binding}
    \newrow{\kwt{fix} \gm{f} \kwt{=} \kwm{\lambda} \kwt{(}\gm{v}\kwt{:}\gm{\tau}\kwt{)} \kwt{->} \gm{\tau'}\kwt{.} \gm{e} \kwt{in} \gm{e'}}{recursive functions\footnotemark[3]}
    \newrow{\kwt{if} \gm{e_1} \kwt{then} \gm{e_2} \kwt{else} \gm{e_3}}{if expression}
    \hline
    \row{l}{\kwt{null}}{unit literal: \kwt{Unit}}
    \newrow{\kwt{true} $|$ \kwt{false}}{boolean literals: \kwt{Bool}}
    \newrow{\dots $|$ \kwm{-}\kwt{2} $|$ \kwm{-}\kwt{1} $|$ \kwt{0} $|$ \kwt{1} $|$ \kwt{2} $|$ \dots}{64-bit signed ints: \kwt{Int}}
    \hline
    \row{p}{\kwt{_} \kwt{:} \gm{\tau}}{discarded pattern}
    \newrow{\gm{v} \kwt{:} \gm{\tau}}{single argument}
    \newrow{\kwt{(}\gm{p_1}, \dots, \gm{p_n}\kwt{)}}{$n$-tuple destructor, $n\geq 2$}
    \hline
    \row{op}{\kwt{\&} $|$ \kwt{|} $|$ \kwt{<} $|$ \kwt{=} $|$ \kwt{>} $|$ \kwt{+} $|$ \kwt{-} $|$ \kwt{*}}{binary operations}
    \hline
  \end{tabular}
\end{figure}
\footnotetext[1]{One can also denote a curried multi-argument function with the syntax \kwm{\lambda} \kwt{(}\gm{v_1}\kwt{:}\gm{t_1}\kwt{)} \gm{\dots} \kwt{(}\gm{v_n}\kwt{:}\gm{t_n}\kwt{)}\kwt{.} \gm{e}, which desugars to $n$ nested lambda expressions. Note that in this case, parentheses are needed around each annotated argument.}
\footnotetext[2]{Similarly, \kwm{\Lambda} \gm{t_1} \gm{\dots} \gm{t_n}\kwt{.} \gm{e} is $n$ nested type abstractions.}
\footnotetext[3]{To define mutually recursive functions, connect the definitions using the keyword \kwt{and}. For example, one can have \kwt{fix} \gm{f_1} \kwt{=} \kwm{\lambda} \kwt{(}\gm{v_1}\kwt{:}\gm{\tau_1}\kwt{)} \kwt{->} \gm{\tau_1'}\kwt{.} \gm{e_1} \kwt{and} \kwt{\dots} \kwt{and} \gm{f_n} \kwt{=} \kwm{\lambda} \kwt{(}\gm{v_n}\kwt{:}\gm{\tau_n}\kwt{)} \kwt{->} \gm{\tau_n'}\kwt{.} \gm{e_n} \kwt{in} \gm{e'}}


\subsection{Types}

\begin{figure}[h]
  \centering
  \begin{tabular}{l p{\ColOne} p{\ColTwo} p{\ColThree} p{\ColFour}}
    \hline
    \row{\tau}{\gm{T}}{type variable (uppercase)}
    \newrow{\kwt{(}\gm{\tau}\kwt{)}}{parenthesized}
    \newrow{\gm{\tau_1} \kwt{->} \gm{\tau_2}}{arrow types}
    \newrow{\kwm{\,\forall} \gm{T}\kwt{.} \gm{\tau}}{universal types}
    \newrow{\gm{\tau_1} \kwm{*} \gm{\dots} \kwm{*} \gm{\tau_n}}{tuple types, $n\geq 2$}
    \newrow{\kwt{Int} $|$ \kwt{Bool} $|$ \kwt{Unit}}{built-in types}
    \hline
  \end{tabular}
\end{figure}

\subsection*{Declarations}

\begin{figure}[h]
  \centering
  \begin{tabular}{l p{\ColOne} p{\ColTwo} p{\ColThree} p{\ColFour}}
    \hline
    \row{\delta}{\kwt{let} \gm{p} \kwt{=} \gm{e} }{declaration}
    \hline
  \end{tabular}
\end{figure}

\subsection*{Alternative syntax}
We can write \kwt{\textbackslash} or \kwt{lambda} instead of \kwm{\lambda}.

We can write \kwt{any} in place of \kwm{\Lambda}.

We can write \kwt{forall} in place of \kwm{\forall}.


\section*{Semantics:}
Call-by-value big step semantics.

When a bound variable is bound again, the new binding takes over.

There is no one-type tuples

Lexical scope.


\end{document}